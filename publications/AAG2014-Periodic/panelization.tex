\documentclass[article.tex]{subfiles}
\begin{document}

\section{Case Study: Hexagonal surface panelization}
\label{sec:panelization}
Here we present the findings of a first case study, where the method
of discrete conformal mappings was utilized for a real-world project.
In this case study, a facade design, important questions concerning
panel layout, similarity and therewith constructability had to be
addressed at the early design stage.

Discrete conformal maps were used, as they allow the designer to
explore alternative surface textures and surface panelizations with
great design flexibility. This distinguishes the method from more
constrained modelling techniques~\cite{Ceccato}. Through the method of
conformal mapping, opposed to naive UV mapping, the density of the
surface panelization varies across the entire surface, yet the shape
of elements does not. This can be used for structural purposes, such
as diagrid layouts, or design driven, such as window distribution. The
optimization of the surface panelization towards multiple criteria
such as edge length and planarity was consequential.

\begin{figure}[t]
  \centering
  \includegraphics[width=0.93\textwidth]{images/henn/overview02.jpg}
  \caption{Rendering of Case Study - A project where the method of conformal mapping was utilized for the facade design.}
  \label{fig:overview02}
\end{figure}

For a commissioned competition entry we tested and developed the
method of periodic conformal mappings. The project, which served as a
case study, was highly constrained, as the architects were asked to
propose an alternative facade design for an existing design proposal
of a multifunctional exhibition center in China, see
Figure~\ref{fig:overview02}. The massing was fixed, but there were 2
alternative massing options (1 single curved and 1 doubly-curved
envelope) to be explored. Also, the client wanted a hexagonal tiling
on the facade but only had a very limited budget of approximately 200
Euro / sqm for the entire facade including sub structure in mind.

These limitations, in combination with the very short timeframe of 2
weeks for the entire redevelopment of the facade including a
feasibility study drove the development of the conformal mapping
method. Especially, since existing solutions such as UV mapping led to
unsatisfactory results producing anisotropic stretch and shear of some
regions in the master surfaces.  Some specific questions that had to
be addressed for each massing option were:

\begin{itemize}
\item How many (different) panels would we need?
%\item If we can group the panels into different types (e.g. planar \&
%  non-planar), how many different groups would we need?
% Is not treated in the article!
\item Can we clad the entire surface with planar tiles?
\item Can we equalize the edge lengths of each hexagon?
\item Can we control the orientation of the panels?
\item Can we achieve a regular pattern with a homogeneous visual
  appearance?
\end{itemize}

In the end, all the above questions were answered/solved.

The first step of development focused on achieving periodicity across
the surface and alignment with the boundary. While the issue of
periodicity directly addressed the last question, it is strongly
related to the others as they could be achieved by successive
optimization steps.

\begin{figure}[t]
  \centering
  \includegraphics[width=0.93\textwidth]{images/henn/panel_construction.jpg}
  \caption{The data for the component-like construction of each panel was derived from the mesh.}
  \label{fig:panel_construction}
\end{figure}

Already during the design phase a fully periodic tiling was
achieved. In a following step the panels were planarized, grouped by
dimension and their edge lengths were equalized. Finally, a control
for the panel orientation based on the tangents of the \nurbs master
surface was implemented. This hexagonal pattern served as a base for
the facade engineering team. Due to the high cost demands, a simple
component system that served as a sub-structure for each panel was
developed, see Figure~\ref{fig:panel_construction}.

Unfortunately the given massing options for the building were not very
challenging in terms of geometry. One massing option was a simple
extrusion and the other had very little distortion. After the
successful submission of the project, we decided to continue the
development and test the method of discrete conformal mappings on more
extreme base geometries.

During these tests a Grasshopper plug-in for \VaryLab,
see~\cite{varylab-web-page} has been developed and refined. We focused
on tiling surfaces with a large distortion/stretch and double
curvature. The main aim was to tile these surfaces without
distorsion. This led to focus on the boundary conditions. The designer
is now able to choose between an aligned mapping where tile-pattern
aligns with the underlying surface boundary. The trade-off being, that
panels need to vary in sizes. Or one chooses a ``homogeneous tiling'',
where all the tiles are the same, but do not align with the
boundary. \VaryLab's numerous optimization algorithms can be applied
and combined with either of the two approaches, see
Figure~\ref{fig:hex_example}. During the development, we realized that
through singularities and special boundary conditions, one is able to
control the density and distribution of the pattern on the surface and
along its boundaries, see Figure~\ref{fig:entrance}.

We also started to look into how patterns can be applied across
multiple surface patches, such that the pattern aligns at the crease
where the patches meet. This would necessitate to develop methods for
more general mappings and the applications of multiple boundary
conditions; A field that is yet to be developed.


\begin{figure}[bt]
  \centering
  \includegraphics[width=0.93\textwidth]{images/henn/detail.jpg}
  \caption{Close-up rendering of facade.}
  \label{fig:detail}
\end{figure}

The collaboration between architect and math department proved to be
very satisfactory for both parties: The architects did provide
specific questions related to real world projects whilst the
mathematicians were able to translate these questions into
mathematical formulas and provided meaningful results that could not
have been achieved alternatively. The common design framework of Rhino
and the basic knowledge of \nurbs geometry and modelling techniques
proved to be of essential importance for the successful collaboration
between the teams.


%Popularization of free-form surfaces in the architectural domain
%opened up a whole new range of challenges related to their further
%subdivision.  A recurring topic is how one could subdivide an
%arbitrary surface into smaller patches while maintaining its aesthetic
%qualities and intentions of the designer.  Surface continuity,
%alignment of the panels, densification of the pattern and integration
%of so-called special zones (such as openings or entrances) are some of
%the reappearing motives.
%
%Architectural practice, especially the conceptual phase of the design
%process, involves “trial and error” methodology, where flexibility and
%ability to test and adjust various ideas is of a great importance.
%Supposingly small changes often mean fundamental
%geometrical/topological modifications and therefore require robust
%modeling tools that allow for fast re-adjustments.
%
%Method of discrete conformal mapping presented in this paper provides
%an alternative solution to architectural tiling, offering wide variety
%of applicable patterns and a lot of control on pattern distribution.
%
%... tbc 
%
%TODO: Agata and Moritz
%
%Describe the architectural tiling aspect of conformal maps. What do
%the different patterns offer? What are the consequences? What is the
%novelty and the new opportunities?

%\subfilebibliography

\end{document}

%%% Local Variables: 
%%% mode: latex
%%% TeX-master: "article"
%%% End: 
