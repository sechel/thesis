\documentclass[article.tex]{subfiles}
\begin{document}

\section{Conclusion}
\label{sec:conclusion}

The collaboration between architect and math department proved to be
very satisfactory for both parties: The architects did provide
specific questions related to real world projects whilst the
mathematicians were able to translate these questions into
mathematical formulas and provided meaningful results that could not
have been achieved alternatively. A common design framework such as Rhino
and the basic knowledge of \nurbs geometry and modelling techniques
proved to be of essential importance for the successful collaboration
between the teams.

As a result of this collaboration, we presented a method for
homogeneous periodic panelization of \nurbs surface geometry of
cylinder type.  A natural future development is the design of suitable
support structures possibly with torsion free nodes. This can be
derived from the panel layout by intersecting the panel
planes. Furthermore the quantization of edge lengths of regular
hexagonal panels is not yet fully explored. A denser distribution of
quantized edge lengths in regions with great edge length variance can
possibly improve the layout and number of different panels. 
%
For a later stage of the project one could add optimization for the
gaps between the panels, which has not been incorporated yet. It would
also be interesting to look at more extreme examples, even if those
might not fit the architectural context.

We also started to look into how patterns can be applied across
multiple surface patches, such that the pattern aligns at the crease
where the patches meet. This would necessitate to develop methods for
more general mappings and the applications of multiple boundary
conditions; A field that is yet to be developed.

While the method presented does require a certain level of expertise
from the user, we see the potential of this post-panelization strategy
of freeform surfaces in the architectural design process: Opposed to
carefully constrained (parametric) modelling approaches this method
allows to realize even distributed, homogenous panelizations on
arbitrary surfaces with cylinder topology at early design stages -
something a lot of designers are interested in.


\subfilebibliography

\end{document}

%%% Local Variables: 
%%% mode: latex
%%% TeX-master: "article"
%%% End: 
