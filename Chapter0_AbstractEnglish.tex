\documentclass[Thesis.tex]{subfiles}
\begin{document}

\section*{Variational Methods for Discrete Surface Parameterization. Applications and Implementation.}
Stefan Sechelmann
\subsection*{Abstract}

The notion of discrete conformal equivalence of triangle meshes was introduced by Luo~\cite{Luo2004:Yamabe} and elaborated in detail by Springborn {\it et al.}~\cite{Springborn2008} and Bobenko {\it et al.}~\cite{BPS2015:dconf}. 
Two combinatorially equivalent euclidean triangle meshes are conformally equivalent if there exist scale factors associated to vertices such that corresponding edge lengths are equal up to multiplication with adjacent scale factors.
The problem of finding a discrete conformal map from a triangulated surface to the plane is formulated in terms of a variational principle using scale factors as variables.

Whereas the theoretical foundations of discrete conformal mappings via discrete conformal equivalence were arranged in previous work, this thesis focuses on the experimental side of the theory. 
In the spirit of discrete differential geometry we look at theorems and constructions from the geometry of Riemann surfaces and find analogous discrete constructions with similar properties.
We investigate the generalization to cyclic polyhedral surfaces, which was only touched upon in previous work.
We also give details on the spherical version of the theory that was put aside previously.

%This thesis is divided into three parts. 

Part~I covers the discrete uniformization of Riemann surfaces via conformal equivalence of cyclic polyhedral surfaces.
We generalize the previously established variational principles accordingly and present a wealth of constructions and examples. 
Taken from smooth differential geometry and translated to the discrete setting we observe properties known from classic theorems. 
This culminates in a theorem about hyperelliptic Riemann surfaces.
A Riemann surface is hyperelliptic if and only if the axes of the uniformization group elements meet in a common point for a suitable basis of the group.
% see Figure~\ref{fig:intro_uniformizations} right.
The tools created in the development of the methods used in this part lay the foundations for the applications presented in Part~III.


In Part~II we present three applications of discrete conformal mappings in the context of architectural geometry. 

In Chapter~2 we calculate regular patterns on architectural facade geometries with a period. 
We investigate how different boundary conditions affect the local behavior of the map.
%The resulting map is used to create new meshes with regular faces such as hexagons.
%We show how these meshes can be optimized towards a facade panel layout that can be fabricated in an efficient manner, see Figure~\ref{fig:intro_applications} left.

In Chapter~3 we exploit the fact that isothermic surfaces possess a conformal curvature line parametrization to create meshes with planar quadrilaterals and touching incircles. 
This problem is formulated as a boundary value problem of discrete conformal maps.
As a result we obtain circle pattern representations of surfaces relevant in the architectural context.
% see Figure~\ref{fig:intro_applications} right.

In Chapter~4 we use discrete conformal maps as initialization for discrete Tschebyshev meshes, i.e., meshes with  edges of equal length. 
Starting from a discrete conformal map, we perform non-linear optimization on quadrilateral meshes.
In doing so we control the curvature and intersection angles of parameter curves in the resulting gridshell construction.

The third part of this work introduces the reader to the software framework built for calculation with discrete surfaces and in particular with discrete conformal mappings.
%, see Chapter~\ref{chp:conformallab} about {\sc ConformalLab}, and discrete surface optimization, Chapter~\ref{chp:varylab} about {\sc VaryLab}. 
%Additionally the author has implemented the software packages {\sc HalfEdge} and {\sc HalfEdgeTools}, see Chapter~\ref{chp:halfedge}, designed as a general tool for the calculation with discrete surfaces. 
%Together with the user interface library {\sc JRWorkspace}, Chapter~\ref{chp:jrworkspace}, it constitutes a flexible framework for the creation of research applications in the context of discrete differential geometry. 

%The digital data accompanying this work includes most of the examples presented in Chapter~\ref{chp:uniformization} and the main files for the architectural applications. 
%Additionally we include the current state of the source code of the programming framework presented in Part~\ref{part:implementation}. 
%The organization of the data is presented in the appendix.

\subfilebibliography
\end{document}

%%% Local Variables:
%%% TeX-master: "Thesis.tex"
%%% End: