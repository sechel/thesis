\documentclass[Thesis.tex]{subfiles}
\begin{document}

\chapter{Preliminaries}

The theory of discrete uniformization presented here is based on the notion of discrete conformal eqivalence of triangle meshes. The Euclidean definition was first considered in \cite{Luo2004}, the variational principle and applications in computer graphics is due to \cite{Springborn2008, OWF2009, Bobenko2010}. The notion of conformal equivalence of non-Euclidean metrics and corresponding variational principles were first defined in \cite{Bobenko2010}. \cite{Guo2011} investigate the gradient flow of this principle.
Most of the material presented here can be found in \cite{BobSechSpr}.

\section{Riemann surfaces}

In this section we introduce basic the notations and constructions of Riemann surfaces that we 
need in order to motivate corresponding discrete notions. 

\begin{definition}{Complex Structure}
...
\end{definition}

\section{Discrete Riemann surfaces}

\begin{figure}
\centering
\scalebox{0.8}{\input{figures/triangles.pdf_t}}
\caption[Discrete surfaces from glued triangles]{Discrete surfaces constructed from glued triangles of constant curvature. Euclidean, hyperbolic, and spherical. Bold edegs are identified to create a cone-like singularity at the vertex.}
\label{fig:surface_triangles}
\end{figure}

\begin{definition}
A \emph{discrete surface} is a collection of triangles equipped with a metric of constant Gaussian curvature and geodesic edges. Triangles are glued along edges to form a surface.
\end{definition}

By glueing triangles equipped with a metric of constant Gaussian curvature we obtain a surface that has constant curvature everywhere except for points where the metric has cone-like singularities (Figure~\ref{fig:surface_triangles}). A discrete surface is called Euclidean for $K=0$, hyperbolic for $K<0$, and spherical if $K>0$.
Remark: In the latter we will use Gaussian curvature and curvature synonymously.
Generically a discrete surface can have boundary components where triangles have not been glued. We consider this case in Section~\ref{sec:planar_domains}. 

A discrete surface consists of vertices, edges, and faces $S=(V, E, F)$. We use single indices for denoting vertices, e.g., $i \in V$, edges are denoted $\it{ij}\in E$, and faces $\it{ijk}\in F$.

\begin{definition}
The map $l:E\to \R$ of triangle edge lengths of a discrete surface is called a \emph{discrete Euclidean, hyperbolic, or spherical metric} respectively.
\end{definition}

As in the smooth theory we define what it means for a metric to be conformally eqivalent to another metric. 

\begin{figure}
\centering
\scalebox{0.5}{\input{figures/equivalence.pdf_t}}
\caption[Euclidean conformal equivalence]{Two Euclidean triangles are discretely conformally equivalent if their edge lengths agree after scaling by logarithmic factors $u$ defined on vertices.}
\label{fig:conformal_equivalence}
\end{figure}

\begin{definition}
\label{def:conformal_equivalence_euclidean}
A discrete Euclidean metric with edge lengths $l$ is \emph{discretely conformally equivalent} to the discrete Euclidean metric $\tilde l$ if there is a function $u:V\to \R$ such that for all edges $\it{ij}\in E$ it is
\begin{equation}
l_{\it{ij}} = e^{\frac{1}{2}(u_i + u_j)}\tilde l_{\it{ij}} \label{eq:euclidean_equivalence}
\end{equation}
\end{definition}

This definition is motivated by the smooth theory of Riemann surfaces where two metrics $g$ and $\tilde g$ on a $2$-manifold $M$ are conformally equivalent if there is a smooth function $u:M\to \R$ with \[g=e^{2u}\tilde g.\]

Every discrete Euclidean metric is discretely conformally equivalent to a corresponding discrete hyperbolic or discrete spherical metric by the following

\begin{definition}
\label{def:conformal_equivalence_general}
A discrete Euclidean metric $l$ and a discrete hyperbolic or discrete spherical metric $\tilde l$ are \emph{discretely conformally equivalent} if for all edges $\it{ij}\in E$
\begin{align}
l_{\it{ij}} &= 2\sinh \frac{\tilde l_{\it{ij}}}{2} \label{eq:hyperbolic_equivalence}\\
l_{\it{ij}} &= 2\sin \frac{\tilde l_{\it{ij}}}{2} \label{eq:spherical_equivalence}
\end{align}
for $\tilde l$ hyperbolic or spherical respectively (see Figure~\ref{fig:conformal_equivalence_sph_hyp}).
\end{definition}

\begin{figure}
\centering
\scalebox{0.5}{\input{figures/spherical.pdf_t}}
\caption[Conformal equivalence of Euclidean and hyperbolic/spherical metrics]{Relations between hyperbolic/spherical lengths $\tilde l$ and corresponding Euclidean edge lengths $l$. See Definition~\ref{def:conformal_equivalence_general}.}
\label{fig:conformal_equivalence_sph_hyp}
\end{figure}

Literally this means that in the spherical case if a triangle is fit onto the sphere the conformally equivalent spherical lengths are the lengths of spherical geodesics connecting the triangle vertices. 
The same intuition holds on the upper sheet of the two-sheeted unit hyperboloid for hyperbolic triangles.

Combining Equations~\ref{eq:euclidean_equivalence} and \ref{eq:hyperbolic_equivalence}/\ref{eq:spherical_equivalence} we can define conformal equivalence of Euclidean and hyperbolic/spherical triangulations via transitivity of equivalence.

With this general notion of conformal equivalence of Euclidean, hyperbolic, and spherical metrics we can now define discrete Riemann surfaces.

\begin{definition}
\label{def:discrete_riemann_surface}
A discrete Riemann surface is an equivalence class of discretely conformally equivalent metrics.
\end{definition}

If one restricts the equivalence class to either Euclidean, hyperbolic, or spherical metrics the length cross-ratio $\mathrm{lcr}_{\it{ij}}$ at edge $\it{ij}$ with opposite vertices $k$ and $m$ is a conformal invariant. Note that this definition depends on the orientation of the surface.

\begin{definition}{Length cross-ratio}
\begin{equation}
	\mathrm{lcr}_{\it{ij}} \colonequals \frac{l_{\it{ik}} l_{\it{jm}}} {l_{\it{mi}} l_{\it{kj}}} \label{eq:length_cross_ratio}
\end{equation}
\end{definition}

\begin{theorem}
\label{thm:moebius_invariance}
Length cross-ratios are invariant under M{\"o}bius transformations of the ambient space.
\end{theorem}


\section{Uniformization}
We can now state the uniformization problem:
\emph{Given a discrete Riemann surface, find a metric of constant curvature
without cone singularities.}

This means that the angles $\alpha_{\it{jk}}^i$ of Euclidean, hyperbolic, or spherical triangles around each vertex  $i\in V$ sum up to $2\pi$ (Figure~\ref{fig:angles_at_vertex}). A discrete surface with a cone-like singularity free metric has constant curvature everywhere. 

As in the smooth case we expect to find a discrete metric with zero Gaussian curvature for tori, a constant negative curvature metric for surfaces with genus $g>1$, and a metric with positive constant curvature for spheres. In the latter we will normalize the curvature of the target spaces to have constant Gaussian curvature $0$, $-1$, or $1$. 

In this work we calculate discrete uniformizations as minimizers of functionals that fit the target geometry. The variational description of the uniformization problem then amounts to finding a critical point of the functional $E(u)$ where

\begin{equation}
\label{eq:gradient_variational}
\frac{\partial E}{\partial u_i} = 2\pi - \sum_{\it{ijk}\ni i} \alpha^i_{\it{jk}}.
\end{equation}  

\begin{figure}
\centering
\scalebox{0.4}{\input{figures/angles.pdf_t}}
\caption[Angles at a vertex]{Angles a a vertex}
\label{fig:angles_at_vertex}
\end{figure}

\subfilebibliography
\end{document}

%%% Local Variables:
%%% TeX-master: "Thesis.tex"
%%% End: