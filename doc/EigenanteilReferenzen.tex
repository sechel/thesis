\documentclass[11pt]{amsart}
\usepackage{german}
\usepackage[utf8]{inputenc}
\usepackage{geometry}
\geometry{a4paper}
\usepackage{graphicx}
\usepackage{amssymb}
\setlength{\parindent}{0pt}

\title{Variational Methods for Discrete Surface Parameterization. Applications and Implementation.}
\author{Stefan Sechelmann}
\date{\today}

\begin{document}
\maketitle

\section*{Erklärung über Vorveröffentlichungen und Eigenanteildarstellung}


\subsection*{Kapitel 1 - Discrete uniformization of Riemann surfaces.}
Das Kapitel \emph{Discrete uniformization of Riemann surfaces} ist zum großen Teil in der Veröffentlichung~\cite{BobSechSpr} enthalten. Es ist in Zusammenarbeit mit Alexander Bobenko und Boris Springborn entstanden.
Die Idee zur Verallgemeinerung der Theorie der diskret konformen Abbildungen auf zyklische Netze stammt von Boris Springborn. Die Darstellung der Theorie (1.1 -- 1.3) ist in Zusammenarbeit entstanden.
Alle Ergebnisse der Abschnitte 1.4 -- 1.8 sind von mir berechnet worden. Insbesondere habe ich die Software zur Berechnung aller Experimente implementiert, siehe Teil~III. Alle Bilder dieser Abschnitte sind von mir erstellt worden. Die Darstellung und Formulierungen sind in Zusammenarbeit entstanden.

\subsection*{Kapitel 2 - Surface panelization using periodic conformal maps.}
Gemeinsame Arbeit mit Thilo Rörig, Agata Kycia und Moritz Fleischmann~\cite{Roerig2014}.
Der theoretische Teil wurde von mir in Zusammenarbeit mit Thilo Rörig entwickelt. Wobei ich maßgeblich an der Entwicklung der verschiedenen Parameterisierungen Anteil hatte. Die Optimierung der Netze stammt von Herrn Rörig. Der praktische Teil (Abschnitt 2.4) stammt von Herrn Fleischmann. Alle Abbildungen bis auf 2.8, 2.9 und 2.10 sind von mir berechnet und gerendert worden.

\subsection*{Kapitel 3 - Quasiisothermic mesh layout.}
Gemeinsame Arbeit mit Thilo Rörig und Alexander Bobenko~\cite{Sechelmann2012}. Ich habe zusätzlich die Abschnitte über isometrische Deformation und das diskrete Ellipsoid hinzugefügt.
Die Theorie und Implementierung zu dieser Arbeit stammt von mir. Die Darstellung ist in Zusammenarbeit entstanden.
Alle Abbildung sind von mir erstellt worden.

\subsection*{Kapitel 4 - Optimization of Regular and Irregular Elastic Gridshells.}
Gemeinsame Arbeit mit Elisa Lafuente Hernández, Thilo R\"{o}rig und Christoph Gengnagel~\cite{Lafuente2012}.
Der theoretische Teil sowie dessen Software-Implementation, Abschnitt 4.2 und 4.3.1, stammen von mir. Die Ergebnisse der Abschnitte 4.3.2, 4.3.3 sowie Abschnitt 4.4 über den Flying Dome stammen von Frau Lafuente Hernández. 

\subsection*{Teil III, Kapitel 5--9 - Implementation.}
Konzept, Inhalt und alle Abbildungen aus Teil~III der Arbeit stammen von mir. Die beschriebene Software ist, soweit nicht anders angegeben, von mir implementiert worden.


\section{Eidesstattliche Erklärung}
Ich versichere, dass die Darstellung des Eigenanteils korrekt ist.


\vspace{2cm}
Berlin, den \today, Stefan Sechelmann


\bibliographystyle{abbrv}
\bibliography{../Thesis}

\end{document}  