\documentclass[11pt]{article}
\usepackage{german}
\usepackage[utf8]{inputenc}
\usepackage{geometry}
\geometry{a4paper}
\usepackage{graphicx}
\usepackage{amssymb}
\setlength{\parindent}{0pt}

\title{Variational Methods for Discrete Surface Parameterization. Applications and Implementation.}
\author{Stefan Sechelmann}
\date{\today}

\begin{document}
\maketitle

\section*{a) Erklärung über Vorveröffentlichungen und Eigenanteildarstellung}
\label{sec:eigenanteil}

\subsection*{Kapitel 1 - Discrete uniformization of Riemann surfaces.}
Vorveröffentlicht in:
\begin{quote}
A.~I.~Bobenko, S.~Sechelmann, and B.~Springborn. Discrete conformal maps: Boundary value problems, circle domains, Fuchsian and Schottky uniformization. In A.~I.~Bobenko, editor, Advances in Discrete Differential Geometry. Springer, 2016.
\end{quote}
Die Darstellung der Verallgemeinerung der Theorie der diskret konformen Abbildungen auf zyklische polyedrische Flächen (1.1 -- 1.3) ist in Zusammenarbeit entstanden.
Alle Ergebnisse der Abschnitte 1.4 -- 1.8 sind von mir berechnet worden. Insbesondere habe ich die Software zur Berechnung aller Experimente implementiert, siehe Teil~III. Alle Bilder dieser Abschnitte sind von mir erstellt worden. Die Darstellung und Formulierungen sind in Zusammenarbeit entstanden.

\subsection*{Kapitel 2 - Surface panelization using periodic conformal maps.}
Vorveröffentlicht in:
\begin{quote}
T.~Rörig, S.~Sechelmann, A.~Kycia, and M.~Fleischmann. Surface panelization using periodic con- formal maps. In P.~Block, J.~Knippers, N.~Mitra, and W.~Wang, editors, Advances in Architectural Geometry 2014, page 365. Springer, 2014.
\end{quote}
Der theoretische Teil wurde von mir in Zusammenarbeit mit Thilo Rörig entwickelt. Wobei ich maßgeblich an der Entwicklung der verschiedenen Parameterisierungen Anteil hatte. 
Alle Abbildungen bis auf 2.8, 2.9 und 2.10 sind von mir berechnet und gerendert worden.

\subsection*{Kapitel 3 - Quasiisothermic mesh layout.}
Vorveröffentlicht in:
\begin{quote}
S.~Sechelmann, T.~Rörig, and A.~I.~Bobenko. Quasiisothermic mesh layout. In L.~Hesselgren, S.~Sharma, J.~Wallner, N.~Baldassini, P.~Bompas, and J.~Raynaud, editors, Advances in Architectural Geometry 2012, pages 243–258. Springer, 2012.
\end{quote}
Die Theorie und Implementierung zu dieser Arbeit stammt von mir. Die Darstellung ist in Zusammenarbeit entstanden. Alle Abbildung sind von mir erstellt worden.

\subsection*{Kapitel 4 - Optimization of Regular and Irregular Elastic Gridshells.}
Vorveröffentlicht in:
\begin{quote}
E.~Lafuente~Hernández, S.~Sechelmann, T.~Rörig, and C.~Gengnagel. Topology optimisation of regular and irregular elastic gridshells by means of a non-linear variational method. In L.~Hesselgren, S.~Sharma, J.~Wallner, N.~Baldassini, P.~Bompas, and J.~Raynaud, editors, Advances in Architectural Geometry 2012, pages 147–160. Springer, 2012.
\end{quote}
Der theoretische Teil sowie dessen Software-Implementation, Abschnitt 4.2 und 4.3.1, stammen von mir. 

\subsection*{Teil III, Kapitel 5--9 - Implementation.}
Konzept, Inhalt und alle Abbildungen aus Teil~III der Arbeit stammen von mir. Die beschriebene Software ist, soweit nicht anders angegeben, von mir implementiert worden.

\section*{b) Erklärung zur Anmeldung der Promotionsabsicht}
Hiermit erkläre ich, dass ich bei keiner anderen Hochschule oder Fakultät früher oder gleichzeitig die Anmeldung einer Promotionsabsicht beantragt habe.

\end{document}  