%!TEX root = ./Thesis.tex
\chapter{Uniformization of discrete Riemann surfaces}

The discrete uniformization theory persented here is based on the notion of discrete conformal eqivalence of triangle meshes. The Euclidean definition was first considered by \cite{Luo2004}, the variational principle and applications in computer graphics is due to \cite{Springborn2008, OWF2009, Bobenko2010}. The notion of conformal equivalence of non-Euclidean metrics and corresponding variational principles were first defined in \cite{Bobenko2010}. \cite{Guo2011} investigate the gradient flow of this principle.
Most of the material presented here can be found in \cite{BobSechSpr}.

\section{Discrete Riemann surfaces}

\begin{figure}
\centering
\scalebox{0.8}{\input{figures/triangles.pdf_t}}
\caption[Discrete surfaces from glued triangles]{Discrete surfaces constructed from glued triangles of constant curvature. Euclidean, hyperbolic, and spherical. Bold edegs are identified to create a cone-like singularity at the vertex.}
\label{fig:surface_triangles}
\end{figure}

\begin{definition}
A \emph{discrete surface} is a collection of triangles equipped with a metric of constant Gaussian curvature and geodesic edges. Triangles are glued along edges to form a surface.
\end{definition}

By glueing triangles equipped with a metric of constant Gaussian curvature we obtain a surface that has constant curvature everywhere except for points where the metric has cone-like singularities (Figure~\ref{fig:surface_triangles}). A discrete surface is called Euclidean for $K=0$, hyperbolic for $K<0$, and spherical if $K>0$.
Remark: In the latter we will use Gaussian curvature and curvature synonymously.
Generically a discrete surface can have boundary components where triangles have not been glued. We consider this case in Section~\ref{sec:planar_domains}. 

A discrete surface consists of vertices, edges, and faces $S=(V, E, F)$. We use single indices for denoting vertices, e.g., $i \in V$, edges are denoted $\it{ij}\in E$, and faces $\it{ijk}\in F$.

\begin{definition}
The map $l:E\to \R$ of triangle edge lengths of a discrete surface is called a \emph{discrete Euclidean, hyperbolic, or spherical metric} respectively.
\end{definition}

As in the smooth theory we define what it means for a metric to be conformally eqivalent to another metric. 

\begin{figure}
\centering
\scalebox{0.5}{\input{figures/equivalence.pdf_t}}
\caption[Euclidean conformal equivalence]{Two Euclidean triangles are discretely conformally equivalent if their edge lengths can be scaled by logarithmic factors $u$ defined on vertices.}
\label{fig:conformal_equivalence}
\end{figure}

\begin{definition}
\label{def:conformal_equivalence_euclidean}
A discrete Euclidean metric with edge lengths $l$ is \emph{discretely conformally equivalent} to the discrete Euclidean metric $\tilde l$ if there is a function $u:V\to \R$ such that for all edges $\it{ij}\in E$ it is
\begin{equation}
l_{\it{ij}} = e^{\frac{1}{2}(u_i + u_j)}\tilde l_{\it{ij}} \label{eq:euclidean_equivalence}
\end{equation}
\end{definition}

This definition is motivated by the smooth theory of Riemann surfaces where two metrics $g$ and $\tilde g$ on a $2$-manifold $M$ are conformally equivalent if there is a smooth function $u:M\to \R$ with \[g=e^{2u}\tilde g.\]

Every discrete Euclidean metric is discretely conformally equivalent to a corresponding discrete hyperbolic or discrete spherical metric by the following

\begin{definition}
\label{def:conformal_equivalence_general}
A discrete Euclidean metric $l$ and a discrete hyperbolic or discrete spherical metric $\tilde l$ are \emph{discretely conformally equivalent} if for all edges $\it{ij}\in E$
\begin{align}
l_{\it{ij}} &= 2\sinh \frac{\tilde l_{\it{ij}}}{2} \label{eq:hyperbolic_equivalence}\\
l_{\it{ij}} &= 2\sin \frac{\tilde l_{\it{ij}}}{2} \label{eq:spherical_equivalence}
\end{align}
for $\tilde l$ hyperbolic or spherical respectively (see Figure~\ref{fig:conformal_equivalence_sph_hyp}).
\end{definition}

\begin{figure}
\centering
\scalebox{0.5}{\input{figures/spherical.pdf_t}}
\caption[Conformal equivalence of Euclidean and hyperbolic/spherical metrics]{Relations between hyperbolic/spherical lengths $\tilde l$ and corresponding Euclidean edge lengths $l$. See Definition~\ref{def:conformal_equivalence_general}.}
\label{fig:conformal_equivalence_sph_hyp}
\end{figure}

Literally this means that in the spherical case if a triangle is fit onto the sphere the conformally equivalent spherical lengths are the lengths of spherical geodesics connecting the triangle vertices. 
The same intuition holds on the upper sheet of the two-sheeted unit hyperboloid for hyperbolic triangles.

Combining Equations~\ref{eq:euclidean_equivalence} and \ref{eq:hyperbolic_equivalence}/\ref{eq:spherical_equivalence} we can define conformal equivalence of Euclidean and hyperbolic/spherical triangulations via transitivity of equivalence.

With this general notion of conformal equivalence of Euclidean, hyperbolic, and spherical metrics we can now define discrete Riemann surfaces.

\begin{definition}
\label{def:discrete_riemann_surface}
A discrete Riemann surface is an equivalence class of discretely conformally equivalent metrics.
\end{definition}

If one restricts the equivalence class to either Euclidean, hyperbolic, or spherical metrics the length cross-ratio $\mathrm{lcr}_{\it{ij}}$ at edge $\it{ij}$ with opposite vertices $k$ and $m$ is a conformal invariant. Note that this definition depends on the orientation of the surface.

\begin{equation}
	\mathrm{lcr}_{\it{ij}} = \frac{l_{\it{ik}} l_{\it{jm}}} {l_{\it{mi}} l_{\it{kj}}} \label{eq:length_cross_ratio}
\end{equation}

\section{Uniformization}
We can now state the uniformization problem:
\emph{Given a discrete Riemann surface, find a metric of constant curvature
without cone singularities.}

This means that the angles $\alpha_{\it{jk}}^i$ of Euclidean, hyperbolic, or spherical triangles around each vertex  $i\in V$ sum up to $2\pi$ (Figure~\ref{fig:angles_at_vertex}). A discrete surface with a cone-like singularity free metric has constant curvature everywhere. 

As in the smooth case we expect to find a discrete metric with zero Gaussian curvature for tori, a constant negative curvature metric for surfaces with genus $g>1$, and a metric with positive constant curvature for spheres. In the latter we will normalize the curvature of the target spaces to have constant Gaussian curvature $0$, $-1$, or $1$. 

In this work we calculate discrete uniformizations as minimizers of functionals that fit the target geometry. The variational description of the uniformization problem then amounts to finding a critical point of the functional $E(u)$ where

\begin{equation}
\label{eq:gradient_variational}
\frac{\partial E}{\partial u_i} = 2\pi - \sum_{\it{ijk}\ni i} \alpha^i_{\it{jk}}.
\end{equation}  

\begin{figure}
\centering
\scalebox{0.4}{\input{figures/angles.pdf_t}}
\caption[Angles at a vertex]{Angles a a vertex}
\label{fig:angles_at_vertex}
\end{figure}

\section{Variational principles for discrete metrics in $\mathbb{E}^2$, $\mathbb{H}^2$, and $\mathbb{S}^2$}

Construction of discrete flat metrics. A discrete Euclidean flat metric is the minimizer of a convex functional.
\begin{eqnarray}
\lambda_{ij} &:=& 2\log l_{ij}\\
\tilde\lambda_{ij} &:=& \lambda_{ij}+u_i+u_j\\
f_{Euc}(u_i, u_j, u_k) &:=& \alpha_i \tilde \lambda_{jk} + \alpha_j \tilde \lambda_{ki} + \alpha_k \tilde \lambda_{ij} + 2\left(\ML(\alpha_i) + \ML(\alpha_j) + \ML(\alpha_k)\right)
\end{eqnarray}

\begin{definition}
\begin{eqnarray}
	E_{Euc}(u) &:=& \sum_{ijk\in F}\left(f_{Euc}(u_i, u_j, u_k) - \frac{\pi}{2}\left(\tilde \lambda_{jk} + \tilde \lambda_{ki} + \tilde \lambda_{ij}\right)\right) + \sum_{i\in V} \Theta_i u_i
\end{eqnarray}
\end{definition}

 This definition and the derivatives can be found in \cite{Bobenko2010}

For the hyperbolic case $\lambda$ and $\tilde\lambda$ are defined as before. Further define
\begin{eqnarray}
	\beta_i &:=& \frac{1}{2} \left(\pi + \alpha_i - \alpha_j - \alpha_k \right)\\
	\beta_j &:=& \frac{1}{2} \left(\pi - \alpha_i + \alpha_j - \alpha_k \right)\\
	\beta_k &:=& \frac{1}{2} \left(\pi - \alpha_i - \alpha_j + \alpha_k \right)\\
	f_{Hyp}(u_i, u_j, u_k) &:=&\beta_i \tilde \lambda_{jk} + \beta_j \tilde \lambda_{ki} + \beta_k \tilde \lambda_{ij}\\ 		
				&&+\ML(\alpha_i) + \ML(\alpha_j) + \ML(\alpha_k) + \ML(\beta_i) + \ML(\beta_j) + \ML(\beta_k)\\
				&&+\ML\left(\frac{1}{2} (\pi - \alpha_i - \alpha_j - \alpha_k)\right)
\end{eqnarray}

\begin{definition}
\begin{eqnarray}
	E_{Hyp}(u) &:=& \sum_{ijk\in F}\left(f_{Hyp}(u_i, u_j, u_k) - \frac{\pi}{2}\left(\tilde \lambda_{jk} + \tilde \lambda_{ki} + \tilde \lambda_{ij}\right)\right) + \sum_{i\in V} \Theta_i u_i
\end{eqnarray}
\end{definition}


\section{Quotient spaces and fundamental domains}

Every Riemann surface $R$ has a universal cover $X$, i.e., a simply connected covering space and a corresponding covering map. A metric of constant curvature on a compact $2$-manifold can be realized as the quotient of the universal cover over a uniformizing group.

Triangulated surfaces of genus $g\geq 2$ without boundary can be equipped with a discretely conformally equivalent flat hyperbolic metric \cite{Bobenko2010}. By flat hyperbolic metric we mean that the edge length are hyperbolic and for any vertex the angle sum is $2\pi$. To realize this metric in the hyperbolic plane e.g. in the Poicar\'e disk model one has to introduce cuts along a basis of the homotopy. This creates a simply connected domain in $\mathbb H^2$. Matching cut paths are realated by a hyperbolic motion i.e. the M\"obius transformations that leave the unit disk invariant (Figure~\ref{fig:axes_of_motion}).

\subsection{The cut-graph and fuchsian groups}
\emph{Want so say here: the number of transformations generated by the mapping of corresponding edges equals the number of path segments in the homotopy-cut-graph. They generate a fuchsian group with \#vertices relations}


\begin{figure}
\centering
\includegraphics[width=0.4\linewidth]{cutCuttedBrezel01}
\caption{Hyperbolic flat metric on a genus $2$ surface and the axes of the associated hyperbolic motions.}
\label{fig:axes_of_motion}
\end{figure}

\subsection{Minimal presentation}
\subsection{Separated handles}
\subsection{Opposite sides identified}
\subsection{Canonical Keen Polygons}

\section{Uniformization of embedded genus $g=0$, $g=1$, and $g>1$ surfaces}

\section{Branched coverings of $\Chat$}
In this section we discuss surfaces that arise as branched coverings of $\hat{\C}$. A Riemann surface that can be represented as a double cover of $\Chat$ is called elliptic for $g=1$ and hyperelliptic for $g>1$~\cite[p.~235]{Jost2007}. 

Let $\lambda_1,\ldots,\lambda_{2g+2}\in \hat{\C}$, $\lambda_i\neq \lambda_j \forall i\neq j.$. The algebraic curve

\begin{equation}
\label{eq:branched_cover}
C=\{(z,w)\in\C^2 \mid w^2 = \prod_{i=1}^{2g+2}(z-\lambda_i)\}
\end{equation}

is a one dimensional complex manifold. A branched double cover of $\Chat$ is the projection $\pi:C\to\C$, $C\ni(z,w)\mapsto z$. A discrete branched cover of $\Chat$ is a triangulation of $\pi(C)$ with vertices at $\lambda_i$.

\subsection{Elliptic curves}
\subsection{The moduli space}
\subsection{Numerical convergence analysis}
\subsection{The modulus of the Wente torus}

\subsection{Construction of hyperelliptic surfaces}
Any hyperelliptic Riemann surface can be expressed as an algebraic curve of the form
\[ w^2 = \prod_{i=1}^{2g+2}(z-\lambda_i) \quad\quad g\geq1,\quad \lambda_i\neq \lambda_j \forall i\neq j.\]
Here $\lambda_i$ are the branch points of the doubly covered Riemann sphere.

\subsection{Weierstrass points on hyperelliptic surfaces}
A hyperelliptic surface comes together with a holomorphic involution $h$ called the hyperelliptic involution. The branch points are fixed points under this transformation. For a hyperelliptic algebraic curve it is $h(\mu, \lambda)=(-\mu, \lambda)$

\subsection{Canonical domains}
\subsection{Lawsons surface}

\section{Uniformization of Schottky data}
\subsection{Images of isometric circles}
\subsection{Hyperelliptic data}

\section{Conformal maps to $\Chat$}
\subsection{Selection of Branch Data}
\subsection{Examples}

\section{Surfaces with boundary}
\label{sec:surfaces_with_boundary}
\subsection{Variation of edge length}
\subsection{Examples}
\begin{figure}
	\centering
	\includegraphics[width=0.3\linewidth]{image/slit_domain/domain_grid.png}
	\includegraphics[width=0.3\linewidth]{image/slit_domain/image_grid.png}
	\caption{Square with symmetric slit to the circle}
	\label{fig:slit_circle}
\end{figure}


\section{Conformal maps of planar domains}
\label{sec:planar_domains}
\subsection{Boundary conditions}
\subsection{Comparison with examples of the Schwarz-Christoffel community}

%%% Local Variables:
%%% TeX-master: "Thesis.tex"
%%% End: