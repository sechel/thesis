\documentclass[Thesis.tex]{subfiles}
\begin{document}

\chapter{Discrete Conformal Maps of Triangle Meshes}

\section{Conformal Equivalence of Triangle Meshes}

The theory of discrete uniformization presented here is based on the notion of discrete conformal eqivalence of triangle meshes. The Euclidean definition was first considered in \cite{Luo2004}, the variational principle and applications in computer graphics is due to \cite{Springborn2008, OWF2009, Bobenko2010}. The notion of conformal equivalence of non-Euclidean metrics and corresponding variational principles were first defined in \cite{Bobenko2010}. \cite{Guo2011} investigate the gradient flow of this principle.

The theory of discrete conformal parameterizations is based on the definition of discrete conformal equivalence of triangle meshes.

\begin{definition}

\end{definition}

\section{Variational Description}

\section{Boundary Conditions}

\section{Conical Singularities}


\subfilebibliography
\end{document}

%%% Local Variables:
%%% TeX-master: "Thesis.tex"
%%% End: