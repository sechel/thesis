\documentclass[Thesis.tex]{subfiles}
\begin{document}
%\setkeys{Gin}{draft=false}
\chapter{{\sc VaryLab} - Discrete surface optimization}
\label{chp:varylab}

\section{Introduction}

{\sc VaryLab} is a software developed at Berlin Institute of Technology by the author, Thilo R\"orig, and others. It is supported by DFG SFB/TRR 109 Discretization in Geometry and Dynamics. It is designed to be an extensible and modular tool for experiments with discrete surfaces in pure mathematics and applications in industrial geometry. {\sc VaryLab} is used to create the result in Chapters~\ref{chp:periodic_conformal_maps}, \marginpar{bullshit!}\ref{chp:quasiisothermic}, and \ref{chp:gridshells}.

In its core {\sc VaryLab} is a solver for non-linear optimization problems on the coordinates of a given 3D discrete surface. That means given a surface $S$ and functionals $f_1,\ldots,f_n:S\to\R$ we (try to) minimize the combined functional

\begin{eqnarray*}
	f(S) = \sum_{i=1}^n \lambda_i f_i(S)
\end{eqnarray*}
where $\lambda_1,\ldots,\lambda_n\in \R$ are user defined weights. Correspondingly the first and second  derivatives of $f_i(S)$ are weighted by $\lambda_i$
\begin{eqnarray*}
	\nabla f(S) = \sum_{i=1}^n \lambda_i \nabla f_i(S), \quad \nabla\nabla f(S) = \sum_{i=1}^n \lambda_i \nabla\nabla f_i(S).
\end{eqnarray*}

{\sc VaryLab} uses the numerical library {\sc PETSc}/{\sc TAO} \cite{petsc-user-ref, petsc-web-page, tao-user-ref} and the corresponding {\sc Java} bindings \cite{jpetsctao-web-page} for computations. To run optimization methods we need at least an implementation of the functional's value. Other methods need gradient or Hessian of the functional. The most important methods are
 
\begin{tabular}{c | c | c}
	$f$ & $f$, $\nabla f$ & $f$, $\nabla f$, $\nabla\nabla f$\\ \hline
	{\tt NM} Nelder-Mead & {\tt LMVM} Limited-Memory, Variable-Metric & {\tt NLS} Newton Line-Search \\
	& {\tt CG} Conjugate Gradient & {\tt NTR} Newton Trust-Region.
\end{tabular}

In {\sc VaryLab} a functional can choose to implement just the value, see Section~\ref{sec:plugin-api}. Additionally it can implement the gradient and the Hessian of $S$. In principle all methods can be used with all functionals even if those do not implement all data needed for the algorithm. {\sc VaryLab} approximates the values of the gradient or the Hessian if they are missing. 

Derivatives and numerical substitutes, 
Boundary Conditions
Numerics
Build-In Functionals
Plug-in facility and extensibility, 
Data Visualization (scalar, vector)
Remeshing

\begin{figure}
\begin{center}
\includegraphics[width=\textwidth]{varylab/varylab_main.png}
\caption{{\sc VaryLab} user interface.}
\label{fig:varylab_main_ui}
\end{center}
\end{figure}

\section{Plug-in API}
\label{sec:plugin-api}

\section{User interface}

\begin{figure}
\begin{center}
\includegraphics[width=0.5\textwidth]{varylab/optimization_plugins.png}\hfill
\begin{minipage}[b]{0.47\linewidth}
\includegraphics[width=\linewidth]{varylab/optimization.png}\\
\includegraphics[width=\linewidth]{varylab/remeshing.png}
\end{minipage}\\
\vskip 0.05cm
\includegraphics[width=\textwidth]{varylab/protocol.png}
\caption{The main user interface panels of {\sc VaryLab}. List of optimization functional plug-ins and their options (left). Main optimization controls with global constraints and minimizer settings (top right). Remeshing ui for different patterns (right middle). Optimization protocol panel (bottom) shows the progress of the optimization for each activated energy.}
\label{default}
\end{center}
\end{figure}


\section{Periodic conformal maps with {\sc VaryLab}}
In this section we describe how the methods of Chapter~\ref{chp:periodic_conformal_maps} are implemented in {\sc VaryLab}. We split the process in two parts. Part one describes the creation of periodic triangle, quad, or hexagonal meshes from an initial unstructured triangle mesh. Part two deals with optimization of panels created from a mesh in part one.

\subsection{Periodic Parameterization}
The parameterization part of the work is carried out via the {\sc ConformalLab} main user interface. 

\begin{itemize}
\item[0] Load a surface with two boundary components. You can map the surface using the three methods described in the article, map to a cylinder (a), polygonal map to a cone of revolution (b), and isometric boundary (c) to an adapted cone of revolution.
\item[1(a)] For the map to the cylinder create a conformal parameterization with straight boundary using the [Discrete Conformal Parameterization] panel and [Quantized Angles/Straight] boundary mode. A cut is introduced automatically to uniquely define the map.\\

\begin{center}
\begin{minipage}{0.9\linewidth}
            \centering
            \resizebox{\linewidth}{!}{
            	\includegraphics{periodic/step00.png}
            	\includegraphics{periodic/step01_surface.png}	
            	\includegraphics{periodic/step02_surface.png}		
            }
            \resizebox{\linewidth}{!}{
                \includegraphics[width=3.2cm]{periodic/step01_ui.png}
                \begin{minipage}[b]{10cm}
                    \includegraphics[width=\linewidth]{periodic/step01a_map.pdf}\\
                    \includegraphics[width=\linewidth]{periodic/step02a_map.pdf}
                \end{minipage}
            }
            \captionof{figure}{Map to cylinder. Start with an automatically cut mesh (top-middle and upper domain image) and modify the cut (top-right and lower domain image) such that the remeshing algorithm can handle the complete boundary.}
            \label{fig:periodic_algorithm1a}
\end{minipage}
\end{center}            

\item[1(b)] For a polygonal map to a cone of revolution, select boundary vertices to specify boundary angles for the domain. Use multiples of $\frac{\pi}{3}$ for triangle/hexagonal panels and multiples of $\frac{\pi}{4}$ for quadrilaterals.

\begin{center}
\begin{minipage}{0.9\linewidth}
            \centering
            \resizebox{0.7\textwidth}{!}{
            	\includegraphics{periodic/step01b_map.pdf}
            	\includegraphics{periodic/step02b_map.pdf}	
            }
            \captionof{figure}{Mapping the surface from Figure~\ref{fig:periodic_algorithm1a} to a hex pattern adapted domain with polygonal boundary curve. Cut orthogonal to the boundary to create a domain that can easily be meshed with boundary aligned hexagons. The domain is identified along the cut via a rotation by $\pi$.}
\end{minipage}
\end{center}

\item[1(c)] Create an isometric boundary and a map to cone of revolution using the [Read Isometric Angles] boundary mode and the [NoCuts] cut strategy. It creates a map to an arbitrary cone of revolution and reads off the resulting boundary angles. Those are set as new boundary conditions (red vertex selection). In a second step choose the [Quantized Angle Periods] boundary mode and the desired quantization for the final map to the pattern adapted cone of revolution.

\begin{center}
\begin{minipage}{0.9\linewidth}
            \centering
            \resizebox{\linewidth}{!}{
            	\includegraphics[width=5cm]{periodic/step01c_map.pdf}
            	\includegraphics[width=5cm]{periodic/step02c_map_hexadapted.pdf}	
            	\includegraphics[width=6cm]{periodic/step02c_surface.png}		
            }
            \captionof{figure}{Map to cylinder with isometric boundary. Create a map with isometric boundary without cutting the mesh (left). Modify the resulting boundary angles to support a periodic pattern (middle). Periodic hexagonal pattern on the surface (right)}
\end{minipage}
\end{center}    

\item[2] Use the [Cut and Glue Texture Domain] command and select [Orthogonal to Boundary] to create a map to a rectangle (a), a map to a polygonal region (b). In the isometric boundary case (c) we do not need a polygon as boundary curve.
\item[3] Select a predefined texture for quadrilateral, triangle, or hexagonal mesh preview from the [Content Appearance->Texture] panel. Adjust the texture scale to a reasonable value. In the isometric case (c) close the period with the texture transformation user interface manually. For cases (a) and (b) periods will be closed automatically by the remeshing algorithm.

\begin{center}
\begin{minipage}{0.9\linewidth}
            \centering
            \resizebox{\linewidth}{!}{
            	\includegraphics[width=5cm]{periodic/step03_quads.png}
            	\includegraphics[width=5cm]{periodic/step03_triangles.png}	
            	\includegraphics[width=1.5cm]{periodic/step03_ui.png}	
            }
            \captionof{figure}{Pattern preview.}
\end{minipage}
\end{center}    

\item[4] Perform remeshing in cases (a) and (b) either for [Boundary Aligned Triangles] or [Boundary Aligned Quads] using the [Surface Remeshing] panel. For non-boundary aligned parameterization (c) use the [Quads With Singularities/Triangles With Singularities] remeshing mode.\\

\begin{center}
\begin{minipage}{0.9\linewidth}
            \centering
            \resizebox{\linewidth}{!}{
            	\includegraphics[width=5cm]{periodic/step04_quads_surface.png}
            	\includegraphics[width=5cm]{periodic/step04_triangles_surface.png}	
            	\includegraphics[width=2cm]{periodic/step04_ui.png}	
            }
            \includegraphics[width=\linewidth]{periodic/step04_quads_map.pdf}
            \includegraphics[width=\linewidth]{periodic/step04_triangles_map.pdf}
            \captionof{figure}{New mesh and domain.}
\end{minipage}
\end{center}    

\item[5] Create a watertight mesh using the [Watertight Mesh Generator] and remove extra edges and vertices. In the isometric case (c) use a combination of [Topology->Stitch] and [Topology->Stitch Cut Path] to remove the cut path from the mesh.
\item[6] For hexagonal mesh creation select from the periodic triangle mesh all centers of hexagons using the [Selection->Lattice] command. The [Remove Vertex and Fill] command then creates a periodic hexagonal mesh.
\begin{center}
\begin{minipage}{0.9\linewidth}
            \centering
            \resizebox{\linewidth}{!}{
            	\includegraphics{periodic/step05.png}
            	\includegraphics{periodic/step06.png}
            	\includegraphics{periodic/step07.png}	
            }
            \captionof{figure}{Creating a periodic hexagonal mesh from a triangle mesh.}
\end{minipage}
\end{center}                
\end{itemize}

\subsection{Panel optimization}

\begin{itemize}
\item[8] [Topology]->[Explode] creates separate faces. Use a [Mean face Edge Length] histogram to show the density of edge lengths. If you want planar panels you should planarize them now. \\
\resizebox{\linewidth}{!}{
	\includegraphics[width=5cm]{periodic/step08_surface.png}
	\includegraphics[width=5cm]{periodic/step08_histogram.png}		
}
\item[9] Equalize the edge lengths per face using the [Springs] Energy and [F-const] option. Use the [Floor] rounding method. Press [Update] to set target lengths per face.\\
\resizebox{\linewidth}{!}{
	\includegraphics[width=13cm]{periodic/step09_surface.png}
	\includegraphics[width=5cm]{periodic/step09_fconst.png}
}
\item[10] From the histogram read off the smallest and largest edge lengths and transfer those into the [Springs] energy ui. Select the [discr.] option and the number of discrete steps.
\item[11] Optimize the surface to consist of a limited number of panel sizes.\\
\resizebox{\linewidth}{!}{
	\includegraphics[width=13cm]{periodic/step11_surface1.png}
	\includegraphics[width=5cm]{periodic/step10_springs.png}
}
\end{itemize}



\section{Quasiisothermic meshes with {\sc VaryLab}}
\setkeys{Gin}{draft=true}
\begin{itemize}
\item[0] Load a genus $0$ surface with one boundary component.
\item[1] Calculate curvature direction estimates on interior vertices to find singularity locations and indices with the [Vector Field->Curvature Vector Fields] command. Visualize directions using the [Halfedge Data Visualization] interface. The data is called, e.g., [Kmax Vec V] for maximum curvature direction with respect to the surface normal.
\item[2] Select singularity vertices and assign corresponding cone angles in the [Selected Nodes] panel of the [Discrete Conformal Parameterization] panel.
\item[3] Calculate curvature direction estimates on boundary edges of the surface, again using the [Vector Field->Curvature Vector Fields] command. Check singularity indices with the [Check Gau{\ss}-Bonnet] button in the [Discrete Conformal Parameterization] panel. It prints the left side of the Gau{\ss}-Bonnet equation to the console. It should give $2\pi$ for a genus $0$ surface in this case.\\
\resizebox{\linewidth}{!}{
\includegraphics[width=5cm]{quasiisothermic/step00.png}
%\includegraphics[width=5cm]{quasiisothermic/step01.png}
\includegraphics[width=5cm]{quasiisothermic/step02_surface.png}
\includegraphics[width=2.6cm]{quasiisothermic/step02_ui.png}
\includegraphics[width=5cm]{quasiisothermic/step03_surface.png}
}
\item[4] Create a discrete conformal parameterization. Use [Conformal Curvature] as boundary mode. Select a quad texture to visualize the parameterization on the surface.
\item[5] Move the texture such that singularities lie either in the middle or at a corner of a quad of the texture. Use the [Texture Remeshing->Transform Texture] command to transform the texture such that two selected vertices lie on $(0,0)$ and $(1,0)$ respectively. This method woks for one or two singularities. We do not implement methods to distort the mapping to match more that two singularities.
\resizebox{\linewidth}{!}{
\includegraphics[width=5cm]{quasiisothermic/step04_surface.png}
\includegraphics[width=4cm]{quasiisothermic/step04_map.pdf}
\includegraphics[width=5cm]{quasiisothermic/step05_surface.png}
}
\item[6] Create a subdivision quad-mesh using the [Surface Remeshing] panel and the [Quads With Singularities] mode. This mode is available in the [Expert] mode of the panel. Press the [Lift/Flat] button to lift the subdivision to the surface.
\item[7] Use the [Texture Remeshing->TextureGeometry] command to extract a quad mesh from the subdivision mesh. Disable the layer of the original mesh as we will continue to work with the quad-mesh.
\resizebox{\linewidth}{!}{
\includegraphics[width=4cm]{quasiisothermic/step06_ui.png}
\includegraphics[width=4cm]{quasiisothermic/step06_flat_rotate.png}
\includegraphics[width=5cm]{quasiisothermic/step06_surface.png}
\includegraphics[width=5cm]{quasiisothermic/step07_surface.png}
}
\item[8] Sew up the path from the boundary to the singularity using the [Topology-> Stitch Cut Path] command. Select the two boundary vertices and a connected edge on the path.
\item[9] Remove extra vertices with the [Texture Remeshing->Collapse 1,2-Valent Vertices] command.
\item[10] Clean up the mesh from extra edges and vertices with the [Selection->Geodesic], [Edit->Remove Edge And Fill], and [Texture Remeshing->Collapse 1,2-Valent Vertices] command.
\resizebox{\linewidth}{!}{
\includegraphics[width=3cm]{quasiisothermic/step08_selection.png}
\includegraphics[width=3cm]{quasiisothermic/step09.png}
\includegraphics[width=3cm]{quasiisothermic/step10_selection.png}
\includegraphics[width=5cm]{quasiisothermic/step10_surface.png}
}
\end{itemize}

\section{Gridshell creation using {\sc VaryLab}}

\subfilebibliography
\end{document}

%%% Local Variables:
%%% TeX-master: "Thesis.tex"
%%% End: