\documentclass[Thesis]{subfiles}
\begin{document}

\chapter{CD Content}
\label{chp:cd_content}

\begin{figure}
\dirtree{%
.1 /\DTcomment{Root folder of the data disk}.
.2 applications.
.3 gridshells\DTcomment{Models and images of Chapter~\ref{chp:gridshells}}.
.3 periodic\DTcomment{Models and images treated in Chapter~\ref{chp:periodic_conformal_maps}}.
.3 quasiisothermic\DTcomment{Models and images of Chapter~\ref{chp:quasiisothermic}}.
.2 data\DTcomment{Example data for Chapter~\ref{chp:uniformization}}.
.3 algorithm.
.3 bear\_torus.
.3 branched\_euclidean\_genus\_2.
.3 branched\_genus\_1.
.3 ....
.2 java\DTcomment{{\sc Java} source files of the listings of Chapter~\ref{chp:jrworkspace}}.
.2 publications\DTcomment{Preprint versions of the original articles of Part~\ref{part:uniformization} and Part~\ref{part:applications}}.
.2 software\DTcomment{Source code and binaries of the software packages described in Part~\ref{part:implementation}}.
.3 jtem.
.4 halfedge\DTcomment{{\sc HalfEdge}}.
.4 halfedgetools\DTcomment{{\sc HalfEdgeTools}}.
.4 jrworkspace\DTcomment{{\sc JRworkspace}}.
.3 conformallab\DTcomment{{\sc ConformalLab}}.
.3 varylab\DTcomment{{\sc VaryLab}}.
.2 Thesis.pdf\DTcomment{This document in PDF file format}.
}
\vspace{0.2cm}
\caption{Directory structure of the data provided with this thesis.}
\label{fig:dir_structure}
\end{figure}

The data disk that accompanies this work contains data and source code. 
See Figure~\ref{fig:dir_structure} for an overview of the directory structure of the data. 
We include the data and images for most of the objects investigated in Chapter~\ref{chp:uniformization}. 
For many of the examples we include more instances of the same experiment varying in genus or discretization resolution.

For example the folder {\tt /data/schottky\_g2} contains data for a Riemann surface of genus $2$ given by Schottky data, see Section~\ref{sec:schottky}.
It contains the files {\tt genus2.xml}, {\tt genus2\_fine2.xml}, and {\tt genus2\_fine2.xml} differing in the number of vertices on the Schottky circles and in the number of vertices approximating the metric in the interior of the fundamental domain.

Data belonging to chapters of the applications part is contained in the respective folders 

\section{Licensing}
All software packages belonging to the {\sc Jtem} (\href{http://www.jtem.de}{www.jtem.de}) collection and the corresponding source code are licensed under the \href{http://opensource.org/licenses/bsd-license.php}{{BSD} 2-{C}lause {L}icense}. This includes the libraries {\sc JRworkspace} (Chapter~\ref{chp:jrworkspace}), {\sc HalfEdge}, and {\sc HalfEdgeTools} (Chapter~\ref{chp:halfedge}).

{\sc ConformalLab} is licensed as follows (copied from creativecommons.org): {\sc ConformalLab} by \href{http://www.sechel.de}{Stefan Sechelmann} is licensed under a \href{http://creativecommons.org/licenses/by-nc/4.0/}{Creative Commons Attribution-NonCommercial 4.0 International License}.

{\sc VaryLab} and its source code does not have a license yet. The source code and binaries are included on the data disk. If you want to use {\sc VaryLab} please contact the author or Thilo R\"orig  or visit the webpage at \url{www.varylab.com}.

\end{document}