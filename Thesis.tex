\documentclass{article}
\usepackage{graphicx}
\usepackage{color}
\usepackage{ams}
\usepackage{amsmath}
\usepackage{amssymb}
\usepackage{amsthm}

% For Milnor's Lobachevsy function $\ML$ 
\usepackage[OT2,T1]{fontenc} 
\newcommand{\ML}{\mbox{\fontencoding{OT2}\fontfamily{wncyr}\fontseries{m}\fontshape{n}\selectfont L}} 

\usepackage[a4paper]{geometry}
\usepackage{fancyhdr}
\pagestyle{fancyplain}
\graphicspath{{image/}}

\newtheorem{definition}{Definition}

\title{Discrete differential geometry of surfaces. Variational principles, algorithms, and implementation}
\author{Stefan Sechelmann}

\begin{document}

\maketitle
\newpage

\tableofcontents

\section{Introduction}
\section{Discrete conformal maps}

\begin{definition}
	Two triangulations $T$ and $\tilde{T}$ are \emph{discrete conformal equivalent} if there is a map $u:V \to \mathbb{R}$ such that for any edge $ij$ it is
	\[l_{ij}=e^{u_i+u_j}\tilde{l}_{ij}\]
\end{definition}

\begin{definition}
	A \emph{discrete flat metric} is a map $l:E\to\mathbb{R_+}$ such that triangle inequalities are satisfied and angle sums around each inner vertex are equal to $2\pi$. 	
\end{definition}


\subsection{Euclidean case}
Construction of discrete flat metrics. A discrete Euclidean flat metric is the minimizer of a convex functional.
\begin{eqnarray}
\lambda_{ij} &:=& 2\log l_{ij}\\
\tilde\lambda_{ij} &:=& \lambda_{ij}+u_i+u_j\\
f(u_i, u_j, u_k) &:=& \alpha_i \tilde \lambda_{jk} + \alpha_j \tilde \lambda_{ki} + \alpha_k \tilde \lambda_{ij} + 2\left(\ML(\alpha_i) + \ML(\alpha_j) + \ML(\alpha k)\right)
\end{eqnarray}

\begin{definition}
\begin{eqnarray}
	E_{Euc}(u) &=& \sum_{ijk\in F}f(u_i, u_j, u_k) - \frac{\pi}{2}\left(\tilde \lambda_{jk} + \tilde \lambda_{ki} + \tilde \lambda_{ij}\right) + \sum_{i\in V} 2\pi u_i
\end{eqnarray}
\end{definition}

\section{Uniformization of elliptic curves}
\subsection{Elliptic Functions}

\subsection{Convergence}

\section{Discrete isothermic parametrizations}

\section{Examples}
\subsection{A discrete ellipsoid}

\end{document}
 