\documentclass[Thesis.tex]{subfiles}
\begin{document}

\chapter{Variational Description}

Much of the material presented here has been published in~\cite{Bobenko2010} and \cite{Springborn2008}. 
Namely the euclidean and hyperbolic functional together with their gradients and hessian matrices. 
We generalize the setting in a way such that not only the scaling factors $u$ are variables of the functional but also the $\lambda$s itself can be varied. 
We contribute the corresponding derivatives. 
In addition to this we also give a spherical version of the functional together with its gradient and hessian matrix. 
As touched upon in \cite{Bobenko2010} letting the $\lambda$s be variables for some edges we are able to create maps to regions bounded by circles. 

\section{Variational principles}

Let $l_{\it ij}$ be the edge lengths of a triangulation. We define
\begin{eqnarray*}
\lambda_{ij} &:=& 2\log l_{ij}\\
\tilde\lambda_{ij} &:=& \lambda_{ij}+u_i+u_j\\
\tilde l_{\it ij} &:=& e^{\frac{1}{2}(\lambda_{\it ij} + u_i + u_j)}
\end{eqnarray*}

\subsection{Euclidean version}

\begin{eqnarray*}
f^{\rm euc}_{\it ijk}(\lambda,u) &:=& \alpha_i \tilde\lambda_{jk} + \alpha_j \tilde\lambda_{ki} + \alpha_k \tilde\lambda_{ij} + 2\left(\ML(\alpha_i) + \ML(\alpha_j) + \ML(\alpha_k)\right)
\end{eqnarray*}

Where angle $\alpha_i$ is the angle opposite to edge $e_{\it jk}$ in the triangle $\it ijk\in F$ with edge lengths $\tilde l_{\it ij}$, $\tilde l_{\it jk}$, and $\tilde l_{\it ki}$.
The solution to the euclidean mapping problem is a critical point of the following functional.

\begin{definition}{Euclidean functional}
\begin{eqnarray}
	E^{\rm euc}(\lambda,u) &:=& \sum_{\it ijk\in F}\left(f^{\rm euc}_{\it ijk}(\lambda,u) - \frac{\pi}{2}\left(\tilde \lambda_{jk} + \tilde \lambda_{ki} + \tilde \lambda_{ij}\right)\right) + \sum_{i\in V} \Theta_i u_i
\end{eqnarray}
\end{definition}

Gradient:
\begin{eqnarray}
	\frac{\partial E^{\rm euc}}{\partial u_i} &=& \Theta_i - \sum_{\it ijk\ni i}\alpha_i \\
	\frac{\partial E^{\rm euc}}{\partial \lambda_{\it ij}} &=& \alpha_k + \alpha_m - \pi
\end{eqnarray}

In the second term the angles $\alpha_k$ and $\alpha_m$ are the angles opposite to the edge $e_{\it ij}$ in the 
triangles on both sides of the edge.

We now derive the hessian matrix from the previously calculated hessian with variables $u$ only:
\[\sum_{i,j\in V}\frac{\partial^2E^{\rm euc}}{\partial u_i \partial u_j} \cdot du_i du_j = \frac{1}{2}\sum_{\it ij \in E} w_{\it ij} (du_i - du_j)^2.\]
Where $w_{\it ij}=\frac{1}{2}(\cot\alpha_k + \cot\alpha_m)$ for triangles $\it kij\in F$ and $\it mji \in F$.
From the definition of $\tilde\lambda_{\it ij} := \lambda_{\it ij}+u_i+u_j$ we obtain
\[du_i-du_j=d\tilde\lambda_{\it ki}-d\tilde\lambda_{\it jk}.\]
If now the original $\lambda$s are variables we get extra terms in the hessian coming from 
\[d\tilde\lambda_{\it ij}=d\lambda_{\it ij}+du_i+du_j\] 
with non-vanishing $d\lambda_{\it ij}$. It is

\begin{eqnarray*}
	\sum_{i,j\in V}\frac{\partial^2E^{\rm euc}}{\partial u_i \partial u_j} \cdot du_i du_j
&+& \sum_{\it ij\in E}\frac{\partial^2E^{\rm euc}}{\partial \lambda_{\it ij} \partial u_i} \cdot \lambda_{\it ij} du_i
+ \sum_{\it ij\in E}\frac{\partial^2E^{\rm euc}}{\partial \lambda_{\it ij} \partial u_j} \cdot \lambda_{\it ij} du_j \\
&=&\frac{1}{2}\sum_{\it ij \in E} w_{\it ij}(d\tilde\lambda_{\it ki}-d\tilde\lambda_{\it jk})^2.\\
&=&\frac{1}{2}\sum_{\it ij \in E} w_{\it ij}(d\lambda_{\it ki}-d\lambda_{\it jk} + du_i - du_j)^2
\end{eqnarray*}
\todo{this looks asymmetric, check, then again every edge is visited twice}

where again $w_{\it ij}=\frac{1}{2}(\cot\alpha_k + \cot\alpha_m)$.

\subsection{Hyperbolic version}

For the hyperbolic case $\lambda$ and $\tilde\lambda$ are defined as before. We define hyperbolic edge lengths as 
\[l^{\rm hyp}_{\it ij} = 2\arsinh(\tilde l_{\it ij}).\]
All angles $\alpha$ are calculated using the hyperbolic edge lengths and hyperbolic trigonometry.
Further define for the triangle $\it ijk\in F$
\begin{eqnarray}
	\beta_i &:=& \frac{1}{2} \left(\pi + \alpha_i - \alpha_j - \alpha_k \right)\\
	\beta_j &:=& \frac{1}{2} \left(\pi - \alpha_i + \alpha_j - \alpha_k \right)\\
	\beta_k &:=& \frac{1}{2} \left(\pi - \alpha_i - \alpha_j + \alpha_k \right)\\
	f^{\rm hyp}_{\it ijk}(\lambda, u) &:=&\beta_i \tilde \lambda_{jk} + \beta_j \tilde \lambda_{ki} + \beta_k \tilde \lambda_{ij}\\ 		
				&&+\ML(\alpha_i) + \ML(\alpha_j) + \ML(\alpha_k) + \ML(\beta_i) + \ML(\beta_j) + \ML(\beta_k)\\
				&&+\ML\left(\frac{1}{2}\left(\pi - \alpha_i - \alpha_j - \alpha_k\right)\right)
\end{eqnarray}

The hyperbolic functional is defined as

\begin{definition}
\begin{eqnarray}
	E^{\rm hyp}(\lambda, u) &:=& \sum_{\it ijk\in F}\left(f^{\rm hyp}_{\it ijk}(\lambda, u) - \frac{\pi}{2}\left(\tilde \lambda_{\it jk} + \tilde \lambda_{\it ki} + \tilde \lambda_{\it ij}\right)\right) + \sum_{i\in V} \Theta_i u_i
\end{eqnarray}
\end{definition}

As in the euclidean case the gradient of $E_{\rm hyp}$ is

\begin{eqnarray}
	\frac{\partial E^{\rm hyp}}{\partial u_i} &=& \Theta_i - \sum_{\it ijk\ni i}\alpha_i \\
	\frac{\partial E^{\rm hyp}}{\partial \lambda_{\it ij}} &=& \alpha_k + \alpha_m - \pi
\end{eqnarray}
\todo{verify derivative with respect to lambda}

The hessian matrix of $E^{\rm hyp}(u)$ if $\lambda$s are fixed is given by

\[\sum_{i,j\in V}\frac{\partial^2E^{\rm hyp}}{\partial u_i \partial u_j} \cdot du_i du_j = 
\tfrac{1}{2}\sum_{\it ij\in E} \left(\cot\beta_k+\cot\beta_m\right)
  \left((du_i-du_j)^2+\tanh^2\left(\tfrac{\tilde l_{\it ij}}{2}\right)(du_i+du_j)^2\right)
 \]
 
 As in the euclidean case we can derive a version with variable $\lambda$s from the fact that 
 \[d\tilde \lambda_{\it ij}=d\lambda_{\it ij}+du_i+du_j.\]
 
 \begin{eqnarray*}
	&&\sum_{i,j\in V}\frac{\partial^2E^{\rm hyp}}{\partial u_i \partial u_j} \cdot du_i du_j
+ \sum_{\it ij\in E}\frac{\partial^2E^{\rm hyp}}{\partial \lambda_{\it ij} \partial u_i} \cdot \lambda_{\it ij} du_i
+ \sum_{\it ij\in E}\frac{\partial^2E^{\rm hyp}}{\partial \lambda_{\it ij} \partial u_j} \cdot \lambda_{\it ij} du_j \\
&=& \tfrac{1}{2}\sum_{\it ij\in E} \left(\cot\beta_k+\cot\beta_m\right)
  \left((d\lambda_{\it ik}-d\lambda_{\it ki} +  du_i-du_j)^2+\tanh^2\left(\tfrac{\tilde l_{\it ij}}{2}\right)(d\lambda_{\it ij} +du_i+du_j)^2\right)
\end{eqnarray*}

\subsection{Spherical version}

In the spherical case it is
\begin{eqnarray}
	f^{\rm sph}(u_i, u_j, u_k) &:=&-\alpha_i u_i - \alpha_j u_j - \alpha_k u_k +\beta_i \tilde \lambda_{jk} + \beta_j \tilde \lambda_{ki} + \beta_k \tilde \lambda_{ij}\\ 		
				&&+\ML(\alpha_i) + \ML(\alpha_j) + \ML(\alpha_k) + \ML(\beta_i) + \ML(\beta_j) + \ML(\beta_k)\\
				&&+\ML\left(\frac{\pi - \alpha_i - \alpha_j - \alpha_k}{2}\right)	
\end{eqnarray}

\begin{definition}
\begin{eqnarray}
	E_{Sph}(u) &:=& \sum_{ijk\in F}\left(f_{Sph}(u_i, u_j, u_k) - \frac{\pi}{2}\left(\tilde \lambda_{jk} + \tilde \lambda_{ki} + \tilde \lambda_{ij}\right)\right) + \sum_{i\in V} \Theta_i u_i
\end{eqnarray}
\end{definition}

The derivatives of all functionals can be derived from the fact that 
\begin{eqnarray}
	\frac{\partial f}{\partial u_i} &=& \alpha_i
\end{eqnarray}
where $\alpha$ is calculated in the respective geometry. From that it follows 
\begin{eqnarray}
	\frac{\partial E}{\partial u_i} &=& \Theta_i - \sum_{ijk\ni i}\alpha_i.
\end{eqnarray}

\subfilebibliography
\end{document}

%%% Local Variables:
%%% TeX-master: "Thesis.tex"
%%% End: