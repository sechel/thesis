\documentclass[Thesis.tex]{subfiles}
\begin{document}

\chapter{Variational Description}

\section{Variational principles for discrete metrics in $\mathbb{E}^2$, $\mathbb{H}^2$, and $\mathbb{S}^2$}

Construction of discrete flat metrics. A discrete Euclidean flat metric is the minimizer of a convex functional.
\begin{eqnarray}
\lambda_{ij} &:=& 2\log l_{ij}\\
\tilde\lambda_{ij} &:=& \lambda_{ij}+u_i+u_j\\
f_{Euc}(u_i, u_j, u_k) &:=& \alpha_i \tilde \lambda_{jk} + \alpha_j \tilde \lambda_{ki} + \alpha_k \tilde \lambda_{ij} + 2\left(\ML(\alpha_i) + \ML(\alpha_j) + \ML(\alpha_k)\right)
\end{eqnarray}
The angles $\alpha$ are calculated from the metric $\tilde{l}$ via corresponding half-angle formulas.

\begin{definition}
\begin{eqnarray}
	E_{Euc}(u) &:=& \sum_{ijk\in F}\left(f_{Euc}(u_i, u_j, u_k) - \frac{\pi}{2}\left(\tilde \lambda_{jk} + \tilde \lambda_{ki} + \tilde \lambda_{ij}\right)\right) + \sum_{i\in V} \Theta_i u_i
\end{eqnarray}
\end{definition}

 This definition and the derivatives can be found in \cite{Bobenko2010}

For the hyperbolic case $\lambda$ and $\tilde\lambda$ are defined as before. Further define
\begin{eqnarray}
	\beta_i &:=& \frac{1}{2} \left(\pi + \alpha_i - \alpha_j - \alpha_k \right)\\
	\beta_j &:=& \frac{1}{2} \left(\pi - \alpha_i + \alpha_j - \alpha_k \right)\\
	\beta_k &:=& \frac{1}{2} \left(\pi - \alpha_i - \alpha_j + \alpha_k \right)\\
	f_{Hyp}(u_i, u_j, u_k) &:=&\beta_i \tilde \lambda_{jk} + \beta_j \tilde \lambda_{ki} + \beta_k \tilde \lambda_{ij}\\ 		
				&&+\ML(\alpha_i) + \ML(\alpha_j) + \ML(\alpha_k) + \ML(\beta_i) + \ML(\beta_j) + \ML(\beta_k)\\
				&&+\ML\left(\frac{\pi - \alpha_i - \alpha_j - \alpha_k}{2}\right)
\end{eqnarray}

\begin{definition}
\begin{eqnarray}
	E_{Hyp}(u) &:=& \sum_{ijk\in F}\left(f_{Hyp}(u_i, u_j, u_k) - \frac{\pi}{2}\left(\tilde \lambda_{jk} + \tilde \lambda_{ki} + \tilde \lambda_{ij}\right)\right) + \sum_{i\in V} \Theta_i u_i
\end{eqnarray}
\end{definition}

In the spherical case it is
\begin{eqnarray}
	f_{Sph}(u_i, u_j, u_k) &:=&-\alpha_i u_i - \alpha_j u_j - \alpha_k u_k +\beta_i \tilde \lambda_{jk} + \beta_j \tilde \lambda_{ki} + \beta_k \tilde \lambda_{ij}\\ 		
				&&+\ML(\alpha_i) + \ML(\alpha_j) + \ML(\alpha_k) + \ML(\beta_i) + \ML(\beta_j) + \ML(\beta_k)\\
				&&+\ML\left(\frac{\pi - \alpha_i - \alpha_j - \alpha_k}{2}\right)	
\end{eqnarray}

\begin{definition}
\begin{eqnarray}
	E_{Sph}(u) &:=& \sum_{ijk\in F}\left(f_{Sph}(u_i, u_j, u_k) - \frac{\pi}{2}\left(\tilde \lambda_{jk} + \tilde \lambda_{ki} + \tilde \lambda_{ij}\right)\right) + \sum_{i\in V} \Theta_i u_i
\end{eqnarray}
\end{definition}

The derivatives of all functionals can be derived from the fact that 
\begin{eqnarray}
	\frac{\partial f}{\partial u_i} &=& \alpha_i
\end{eqnarray}
where $\alpha$ is calculated in the respective geometry. From that it follows 
\begin{eqnarray}
	\frac{\partial E}{\partial u_i} &=& \Theta_i - \sum_{ijk\ni i}\alpha_i.
\end{eqnarray}

\subfilebibliography
\end{document}

%%% Local Variables:
%%% TeX-master: "Thesis.tex"
%%% End: