%!TEX root = ./Thesis.tex
\chapter{Examples}

\section{Branched coverings of $\Chat$}
In this section we discuss Riemann surfaces that arise as branched coverings of $\hat{\C}$. A Riemann surface that can be represented as a double cover of $\Chat$ is called elliptic for $g=1$ and hyperelliptic for $g>1$~\cite[p.~235]{Jost2007}. 

Let $\lambda_1,\ldots,\lambda_{2g+2}\in \hat{\C}$, $\lambda_i\neq \lambda_j \forall i\neq j.$. The algebraic curve

\begin{equation}
\label{eq:branched_cover}
C=\{(z, \mu)\in\C^2 \mid \mu^2 = \prod_{i=1}^{2g+2}(z-\lambda_i)\}
\end{equation}

is a one dimensional complex manifold. A branched double cover of $\Chat$ is the projection $\pi:C\to\C$, $C\ni(z,\mu)\mapsto z$. A discrete branched cover of $\Chat$ is a triangulation of $\pi(C)$ with vertices at $\lambda_i$.

\subsection{Elliptic curves}
\subsection{The moduli space}
\subsection{Numerical convergence analysis}

\subsection{Construction of hyperelliptic surfaces}
Any hyperelliptic Riemann surface can be expressed as an algebraic curve of the form
\[ w^2 = \prod_{i=1}^{2g+2}(z-\lambda_i) \quad\quad g\geq1,\quad \lambda_i\neq \lambda_j \forall i\neq j.\]
Here $\lambda_i$ are the branch points of the doubly covered Riemann sphere.

\subsection{Weierstrass points on hyperelliptic surfaces}
A hyperelliptic surface comes together with a holomorphic involution $h$ called the hyperelliptic involution. The branch points are fixed points under this transformation. For a hyperelliptic algebraic curve it is $h(\mu, \lambda)=(-\mu, \lambda)$

\subsection{Canonical domains}
\subsection{The Wente torus}
\subsection{Lawsons surface}



\section{Schottky to Fuchsian Uniformization}
Let $C_1,C'_1\ldots,C_g,C'_g$ be disjoint circles in $\hat{\C}$. A classical Schottky group $G$ is a Kleinian group
with generators $\sigma_1,\ldots,\sigma_g$ satisfying
\[\frac{\sigma_i z - B_i}{\sigma_i z - A_i} = \mu_i \frac{z - B_i}{z - A_i}, \quad\quad\quad 0 < \left|\mu_i\right|<1,\]
where $\sigma_i$ maps the exterior of $C_i$ onto the interior of $C'_i$. The points $A_i$ and $B_i$ lie inside the circles $C_i$ and $B_i$ respectively \cite{bobenko2011riemann}. Figure~\ref{fig:schottky_group} illustrates this construction.
The quotient space $\Chat/G$ is a Riemann surface of genus $g$. The group $G$ is called the uniformizing
Schottky group

\begin{figure}
	\centering
	\scalebox{1.0}{\input{figures/schottky2.pdf_t}}
	\caption{Schottky group generating a Riemann surface of genus $2$. The point at infinity is not contained in 
any of the circles. The parameters $A$ and $B$ lie inside the cirlces by definition.}
	\label{fig:schottky_group}
\end{figure}

Given a Riemann surface of genus $g$ and a cassical Schottky uniformization, we calculate a corresponding 
Fuchsian uniformization using discrete conformal equivalence of spherical and hyperbolic triangle meshes.
A fundamental domain of the Schottky uniformization is the Riemann sphere without the interior of the circles.

\subsection{Discretization}

We start with a classical Schottky group $G$. Let $\sigma_1,\ldots,\sigma_g$ be the generators of $G$. We choose circles around the fixed points of the

The triangulation of a fundamental domain has to respect the mappings between the circles. Vertices on identified 
circles have to be identified as well. It is however not clear a priori what the conformal structure of a triagulated 
fundamental domain is. The edge length on identified circles differ since there is a M{\" o}bius transformation 
between them. 

The idea to define the conformal structure is to use length cross-ratios instead of edge lengths. Both definitions are equivalent and define the conformal structure of a discrete Riemann surface. So given length cross ratios on edges we can calculate representative edge lengths in the discrete conformal class of the surface and, more obvious, vice versa. On boundary edges of the fundamental domain the length cross ratios agree since length-cross ratios are M{\" o}bius invariant.

\begin{figure}
	\centering
	\includegraphics[width=0.4\linewidth]{image/schottky/schottky_domain2.pdf}
	\includegraphics[width=0.4\linewidth]{image/schottky/schottky01.pdf}
	\caption{Fuchsian uniformization of a Riemann surface (right) given by Schottky data (left).}
	\label{fig:fuchsian_to_schottky}
\end{figure}



\subsection{Images of isometric circles}
\subsection{Hyperelliptic data}

\section{Conformal maps to $\Chat$}
\subsection{Selection of Branch Data}
\subsection{Examples}

\section{Surfaces with boundary}
\label{sec:surfaces_with_boundary}
\subsection{Variation of edge length}
\subsection{Examples}
\begin{figure}
\centering
\resizebox{\textwidth}{!}{
\includegraphics[width=0.5\linewidth]{image/slit_domain/domain_grid.png}
\includegraphics[width=0.5\linewidth]{image/slit_domain/image_grid.png}
}
\caption{Square with symmetric slit to the circle}
\label{fig:slit_circle}
\end{figure}


\section{Conformal maps of planar domains}
\label{sec:planar_domains}
\subsection{Boundary conditions}
\subsection{Comparison with examples of the Schwarz-Christoffel community}


%%% Local Variables:
%%% TeX-master: "Thesis.tex"
%%% End: